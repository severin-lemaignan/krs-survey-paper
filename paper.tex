%\documentclass[twoside,a4paper]{report}
\documentclass[a4paper]{article}

\usepackage[utf8]{inputenc}

\usepackage{hyperref}
\usepackage{graphicx}
\usepackage{tabularx}
\usepackage{supertabular}
\usepackage{pdflscape} %% Used for very big table
\usepackage{moreverb}
\usepackage[table]{xcolor}
\usepackage{listings}
\usepackage{fancyhdr}
\usepackage{fixme}

\definecolor{codegray}{gray}{.95}
\definecolor{palegray}{gray}{.65}
\lstset{language=C,
	numbers=left,
	tabsize=2,
	basicstyle=\scriptsize,
	stringstyle=\textrm,
	showstringspaces=false,
	frame=none,
	xleftmargin=3pt,
	backgroundcolor=\color{codegray}}

\title{Current State of Knowledge Representation Systems for Robotics}
\author{Séverin Lemaignan, Moritz Tenorth}

\begin{document}

\maketitle
\tableofcontents

\section{Introduction}
\label{sect|intro}

\subsection{Definition of inclusion criteria}

We have setup the following criteria to select the knowledge representation
systems to include in this survey.

Those systems must:

\begin{itemize}
	\item  Run on \emph{service robot} (robots that interact with objects in a
	semantic-rich environment),

	\item  Ground their knowledge in the physical world (physically embedded),
	\begin{itemize}
		\item  Able to \emph{resolve entities}
		\item  Able to automatically create new object instances
	\end{itemize}

	\item  Be able to merge different knowledge modalities,
	\item  Be endowed with on-line, dynamic reasoning (not just a static
	dictionary)

\end{itemize}

\subsection{Methodology}
\label{sect|methodology}


\fxnote{Mention here that each team was proposed to complete/amend the description of their
system}
	
\section{Comparison criteria}
\label{sect|comparison-criteria}

\subsection{Intrinsic features}
\label{sect|intrinsic-features}

\begin{itemize}
	\item  Expressiveness
	\begin{itemize}
		\item  Which logic formalism (DL, 2nd order\ldots{})
		\item  OWA/CWA
		\item  Representation of uncertainty
		\item  (non) monotonic reasoning
		\item  General knowledge + exception to this knowledge (blue sky/white sky)
		\item  Microtheories
		\item  Lazy evaluation
		\item  Presupposition accomodation (ability to represent facts that are only verbally hinted, but not grounded into perception)
	\end{itemize}

	\item  Explicit modeling of context of knowledge / domain of validity
	\item  Representation of change (\emph{"The pancake dough disappears into a pancake"})
	\item  Support for learning
	\item  Support for introspection?
	\item  Support for prediction/projection?
\end{itemize}

\subsection{Integration in a larger robotic architecture}

\begin{itemize}
	\item  Knowledge acquisition possible\ldots{}
	\begin{itemize}
		\item  \ldots{}through interaction? How?
		\item  \ldots{}through the Web? How?
		\item  \ldots{}through learning? How?
		\item  \ldots{}through perception? How?
	\end{itemize}

	\item  Integration with supervision
	\begin{itemize}
		\item  Events
		\item  Relation to task planning
	\end{itemize}

	\item  Scalability and responsiveness
\end{itemize}

\subsection{Underlying knowledge model}

\begin{itemize}
	\item  Which underlying knowledge (\emph{common-sense}, \emph{upper knowledge}\ldots{})
	\begin{itemize}
		\item  top-down approach?
	\end{itemize}

\end{itemize}

\section{Surveyed systems}

Table \ref{table|surveyed-systems} presents the fifteen knowledge representation systems surveyed
in this article.

This section briefly presents each of them.

\begin{landscape}
\begin{table}
\begin{center}
\rowcolors{2}{lightgray}{codegray}

\begin{tabularx}{\textwidth}{llp{2cm}p{4cm}p{2.3cm}p{2cm}p{2cm}}
\hiderowcolors
{\bf Name} & {\bf Authors (Institution)} & {\bf Surveyed version} & {\bf Project homepage} & {\bf Programming language} & {\bf Knowledge model} & {\bf Reasoner} \\
\hline
\showrowcolors
{\sc KnowRob} & Tenorth (TU Munich) & & & {\sc Prolog} & {\sc Prolog} + OWL-DL & Custom ({\sc Prolog}) \\
ORO & Lemaignan (LAAS-CNRS) & 0.8.0 & \url{oro.openrobots.org} & {\sc Java} & OWL-DL ({\sc Jena}) & {\sc Pellet} \\
PEIS Ecology & Saffiotti, Coradeschi \\
 & Kollar, Tellex \\
OMKRF \\
 & Roy, Mavridis \\
 & Wrighteagle \\
 & Vincze \\
 & (Bielfeld) \\
ARMAR & Schmidt-Rohr (Karlsruhe) \\
 & (Saarbrücken) \\
 & Hertzberg (Osnabrück) \\
 & (DFKI Bremen) \\
 (based on {\sc KnowRob} & (JSK) \\
DY-KNOW & Heitnz, Dowerty \\

\hline

\end{tabularx}
\end{center}
\caption{List of surveyed systems}
\label{table|surveyed-systems}
\end{table}
\end{landscape}

%\begin{itemize}
%	\item  Saffiotti/Coradeschi
%	\begin{itemize}
%		\item  Grounding, anchoring
%		\item  \emph{PEIS Ecology} (more "SmartHome")
%	\end{itemize}
%
%	\item  Tom Kollar/Stefanie Tellex
%	\begin{itemize}
%		\item  Dialog grounding
%	\end{itemize}
%
%	\item  OMKRF
%	\item  Deb Roy/Nikolaos Mavridis
%	\item  Wrighteagle RoboCup team
%	\item  Markus Vincze (Vienna)
%	\item  Bielefeld
%	\item  Karlsruhe
%	\begin{itemize}
%		\item  ARMAR
%		\item  Sven Schmidt-Rohr
%	\end{itemize}
%
%	\item  Saarbrücken (SemanticMap)
%	\item  Joachim Hertzberg (Osnabrueck)
%	\item  DFKI Bremen
%	\begin{itemize}
%		\item  more "SmartHome"
%	\end{itemize}
%
%	\item  JSK
%	\begin{itemize}
%		\item  KnowRob + physics simulator + planning
%	\end{itemize}
%
%	\item  Frederik Heitnz, Patrick Dowerty
%	\begin{itemize}
%		\item  \emph{DY-KNOW} (stream-based knowledge bases)
%	\end{itemize}
%
%\end{itemize}
%

\section{Conclusion}
\label{sect|conclusion}

\subsection{Main groups}

\subsection{What is not successfully covered by current systems}

\subsection{Future challenges}
\label{sect|future-challenges}

%%%%%%%%%%%%%%%%%%%%%%%%%%%%%%%%%%%%%%%%%%%%%%%%%%%%%%%%%%%%%%%%%%%%%%%%%% 
\section*{Acknowledgements} 

%%%%%%%%%%%%%%%%%%%%%%%%%%%%%%%%%%%%%%%%%%%%%%%%%%%%%%%%%%%%%%%%%%%%%%%%%% 

\bibliographystyle{spbasic}
\bibliography{biblio}


\end{document}
