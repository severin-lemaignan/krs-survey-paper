%\documentclass[twoside,a4paper]{report}
\documentclass[twocolumn,a4paper]{article}

\usepackage[utf8]{inputenc}

\usepackage{hyperref}
\usepackage{graphicx}
\usepackage{tabularx}
\usepackage{supertabular}
\usepackage{pdflscape} %% Used for very big table
\usepackage{moreverb}
\usepackage{color}
\usepackage{listings}
\usepackage{fancyhdr}


\definecolor{codegray}{gray}{.95}
\lstset{language=C,
	numbers=left,
	tabsize=2,
	basicstyle=\scriptsize,
	stringstyle=\textrm,
	showstringspaces=false,
	frame=none,
	xleftmargin=3pt,
	backgroundcolor=\color{codegray}}

\hypersetup{
pdftitle = {Survey on Knowledge Representation Systems for Robotics},
pdfauthor = {Séverin Lemaignan, Moritz Tenorth},
pdfkeywords = {cognitive robotics, knowledge representation},
}
\title{Survey on Knowledge Representation Systems for Robotics}
\author{Séverin Lemaignan, Moritz Tenorth}

\begin{document}

\maketitle
\tableofcontents

\section{Which systems do we survey?}

Systems that:


\begin{itemize}
	\item  Run on \emph{service robot} (robots that interact with objects in a semantic-rich environment),
	\item  Ground their knowledge in the physical world (physically embedded),
	\begin{itemize}
		\item  Able to \emph{resolve entities}
		\item  Able to automatically create new object instances
	\end{itemize}

	\item  Are able to merge different knowledge modalities,
	\item  Are endowed with on-line, dynamic reasoning (not just a static dictionary)
\end{itemize}

\section{Criteria, capabilities that are compared}

\subsection{Intrinsic features}

\begin{itemize}
	\item  Expressiveness
	\begin{itemize}
		\item  Which logic formalism (DL, 2nd order\ldots{})
		\item  OWA/CWA
		\item  Representation of uncertainty
		\item  (non) monotonic reasoning
		\item  General knowledge + exception to this knowledge (blue sky/white sky)
		\item  Microtheories
		\item  Lazy evaluation
		\item  Presupposition accomodation (ability to represent facts that are only verbally hinted, but not grounded into perception)
	\end{itemize}

	\item  Explicit modeling of context of knowledge / domain of validity
	\item  Representation of change (\emph{"The pancake dough disappears into a pancake"})
	\item  Support for learning
	\item  Support for introspection?
	\item  Support for prediction/projection?
\end{itemize}

\subsection{Integration in robotic architecture}

\begin{itemize}
	\item  Knowledge acquisition possible\ldots{}
	\begin{itemize}
		\item  \ldots{}through interaction? How?
		\item  \ldots{}through the Web? How?
		\item  \ldots{}through learning? How?
		\item  \ldots{}through perception? How?
	\end{itemize}

	\item  Integration with supervision
	\begin{itemize}
		\item  Events
		\item  Relation to task planning
	\end{itemize}

	\item  Scalability and responsiveness
\end{itemize}

\subsection{Knowledge model}

\begin{itemize}
	\item  Which underlying knowledge (\emph{common-sense}, \emph{upper knowledge}\ldots{})
	\begin{itemize}
		\item  top-down approach?
	\end{itemize}

\end{itemize}

\section{Major players}

\begin{itemize}
	\item  Saffiotti/Coradeschi
	\begin{itemize}
		\item  Grounding, anchoring
		\item  \emph{PEIS Ecology} (more "SmartHome")
	\end{itemize}

	\item  Tom Kollar/Stefanie Tellex
	\begin{itemize}
		\item  Dialog grounding
	\end{itemize}

	\item  OMKRF
	\item  Deb Roy/Nikolaos Mavridis
	\item  Wrighteagle RoboCup team
	\item  Markus Vincze (Vienna)
	\item  Bielefeld
	\item  Karlsruhe
	\begin{itemize}
		\item  ARMAR
		\item  Sven Schmidt-Rohr
	\end{itemize}

	\item  Saarbrücken (SemanticMap)
	\item  Joachim Hertzberg (Osnabrueck)
	\item  DFKI Bremen
	\begin{itemize}
		\item  more "SmartHome"
	\end{itemize}

	\item  JSK
	\begin{itemize}
		\item  KnowRob + physics simulator + planning
	\end{itemize}

	\item  Frederik Heitnz, Patrick Dowerty
	\begin{itemize}
		\item  \emph{DY-KNOW} (stream-based knowledge bases)
	\end{itemize}

\end{itemize}
\end{document}
