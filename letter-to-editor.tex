\documentclass{letter}
\usepackage{letterbib}
\usepackage[utf8]{inputenc}
\usepackage{hyperref}
\usepackage{booktabs}
\usepackage{color}


\signature{Séverin Lemaignan, Moritz Tenorth}
\begin{document}
\newcommand{\todo}[1]{\textcolor{red}{\textbf{TODO:} #1}}

\begin{letter}{Editor-in-Chief \\ Frank Park \\ Seoul National University}
\opening{Dear Editor-in-Chief,}

As recommended on the website of the \emph{Transactions on Robotics}, we contact
you to bring to your attention a survey proposal on Knowledge Representation \&
Reasoning systems for cognitive robotics.

Those systems or frameworks have rapidly grown in number over the last few
years: Since Brachman's and Levesque's book~\cite{brachman2004knowledge} in 
2004, dozens of systems have emerged to fulfil the need for explicit, symbolic (and
sub-symbolic) \emph{representation} and \emph{reasoning} on knowledge in
cognitive architectures for robots.\todo{MT: why do we survey only nine if there are dozens?}

Because these systems have explored many different paths, from building upon purely
logic languages to proposing distributed, diffuse knowledge representation
schemes; because more and more research groups that are looking for increased
robot autonomy look into the implementation of their own framework for knowledge
representation; because, also, no previous survey of these tools has ever been
published to the best of our knowledge, we propose to write such a review of the
existing literature, starting with brief typology of main features to be found
in those systems, followed by a systematic review of the existing approaches.

This survey would be a joint work by Moritz Tenorth from the University of
Bremen, Germany (Michael Beetz's group) and Séverin Lemaignan, currently working
at EPFL, Switzerland and previously with Rachid Alami's group, LAAS-CNRS, France.

Moritz Tenorth is the author and main developer of the well-known {\sc KnowRob}
suite for knowledge management for service robots. This framework has been
deployed on many different robots, ranging from Willow Garage's PR2 to several
custom manipulation platforms, and is being used in several research groups 
world-wide. His publications on this project have been cited over 800 times 
and he is regarded as an expert in this field.

Séverin Lemaignan is the author of the {\sc ORO} knowledge base that has
received special attention in the human-robot interaction community for being
one of the first knowledge representation and reasoning system that is explicitly
taking the cognitive models of humans interacting with the robot into account. 
His platform has also been deployed on several robots in at least five different
laboratories. Séverin Lemaignan is regularly invited for talks on cognitive
architectures.

For this survey, we plan to consider the following systems:

\begin{tabular}{p{2.8cm}p{8cm}p{1.5cm}}
\toprule
{\bf Project} & {\bf Authors (Institution)} & Main references \\
\midrule
ARMAR/Tapas &  Holzapfel, Waibel \par (KIT Karlsruhe) &  \cite{Holzapfel2008}\\
CAST Proxies &  Wyatt, Hawes, Jacobsson, Kruijff (Brimingham Univ., DFKI Saarbrücken) &  \cite{Jacobsson2008} \\
GOLOG & Levesque (Toronto Univ.) & \cite{levesque1997golog} \\
GSM &  Mavridis, Roy \par (MIT MediaLab) & \cite{Mavridis2006} \\
Ke Jia Project & Chen et al. \par (Univ. of Science and Technology of China) &  \cite{Chen2010} \\
{\sc KnowRob} &  Tenorth, Beetz \par (TU Munich) &   \cite{Tenorth2009a} \\
NKRL &  Zarri et al. \par (Paris Est Créteil Univ.) &  \cite{Sabri2011} \\
%OBOC & KRS & Mendoza & & & & & \cite{Mendoza2005} \\
ORO &  Lemaignan, Alami \par (LAAS-CNRS) &  \cite{Lemaignan2010} \\
OUR-K/OMRKF &  Lim, Suh et al. \par (Hanyang Univ.) &   \cite{Lim2011, Suh2007} \\
PEIS KR\&R &  Daoutis, Coradeshi, Loutfi, Saffiotti \par (Örebro Univ.) &  \cite{Daoutis2009} \\
% & & Varadarajan, Vincze \par (TU Wien) & & & & & \cite{Varadarajan2011} \\ % -> affordances, but no implementation on a robot
% & & Kaelbling, Lozano-Pérez \par (MIT CSAIL) & & & & & \cite{Kaelbling2011} \\ % -> mostly planning under uncertainty
% & & Hertzberg (Osnabrück Univ.) \\ % -> affordances, semantic mapping
% (based on {\sc KnowRob} & & (JSK) \\

\bottomrule

\end{tabular}



We hope that you may consider our proposal as interesting and relevant for a 
future issue of \emph{Transaction on Robotics}. If you require any additional 
information, please feel free to contact us.

\closing{Yours Sincerly,}

\ps{Séverin Lemaignan is with the CHILI Lab, EPFL, Lausanne, Switzerland.\\ 
    Moritz Tenorth is with the Institute for Artificial Intelligence, Universität Bremen, Germany.}

%\url{http://www.gnu.org/copyleft/fdl.html}.}
%\encl{Copyright permission form}

\end{letter}

\bibliographystyle{ieeetr}
\bibliography{biblio}


\end{document}
