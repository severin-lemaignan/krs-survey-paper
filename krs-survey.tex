%\documentclass[twoside,a4paper]{report}
\documentclass[a4paper]{article}

\usepackage[utf8]{inputenc}

\usepackage{hyperref}
\usepackage{graphicx}
\usepackage{tabularx}
\usepackage{supertabular}
\usepackage{pdflscape} %% Used for very big table
\usepackage{moreverb}
\usepackage[table]{xcolor}
\usepackage{listings}
\usepackage{fancyhdr}
\usepackage[draft]{fixme}

\definecolor{codegray}{gray}{.95}
\definecolor{palegray}{gray}{.65}
\lstset{language=C,
	numbers=left,
	tabsize=2,
	basicstyle=\scriptsize,
	stringstyle=\textrm,
	showstringspaces=false,
	frame=none,
	xleftmargin=3pt,
	backgroundcolor=\color{codegray}}

\title{Current State of Knowledge Representation Systems for Robotics}
\author{Séverin Lemaignan, Moritz Tenorth}

\graphicspath{{figs/}}

\newcommand{\ie}{{\textit{i.e.~}}}
\newcommand{\cf}{{\textit{cf~}}}
\newcommand{\eg}{{\textit{e.g.~}}}

\begin{document}

\maketitle
\tableofcontents

%%%%%%%%%%%%%%%%%%%%%%%%%%%%%%%%%%%%%%%%%%%%%%%%%%%%%%%%%%%%%%%%%%%%%%%%%%%%%%%%%%%%%%%%%%%
\section{Introduction}
\label{sect|intro}

The idea of \emph{Cognitive robotics} was coined in the early 1990s by Reiter.
In a chapter on that subject in \emph{Fondations of Artifical
Intelligence}~\cite{Levesque2007}, Levesque reminds about the manifesto they
wrote together in 1998:

\begin{quotation}

	Central to this effort is to develop an understanding of the relationship
	between the knowledge, the perception, and the action of such a robot. The
	sorts of questions we want to be able to answer are

	\begin{itemize} 
	
		\item to execute a program, what information does a robot need to have
		at the outset vs. the information that it can acquire \emph{en route}
		by perceptual means?

		\item what does the robot need to know about its environment vs. what
		need only be known by the designer?

		\item when should a robot use perception to find out if something is
		true as opposed to reasoning about what it knows was true in the past?

		\item when should the inner workings of an action be available to the
		robot for reasoning and when should the action be considered primitive
		or atomic?

	\end{itemize}

	and so on. With respect to robotics, our goal (like that of many in AI) is
	\emph{high-level robotic control}: develop a system that is capable of
	generating actions in the world that are appropriate as a function of some
	current set of beliefs and desires.

\end{quotation}

This survey focuses on the knowledge issue: we aims at establishing the current
landscape of approaches to the knowledge representation problem in the research
community.

We have identified over ten projects that are designed for explicit knowledge
manipulation within a robot. By knowledge manipulation, we mainly mean:

\begin{itemize}

	\item symbolic representation of assertions, be it static statements on the
	world or spatio-temporal events,

	\item reasoning that this representation enables,

\end{itemize}

We will also try, for each project, to make clear

\begin{itemize}

	\item how new facts can be acquired, \ie how informations from perception
	or interaction are turned into knowledge,

	\item how, in return, symbolic concept are \emph{anchored} into the robot
	sensori-motor space,

	\item how the knowledge representation system integrates with the
	decisional layers of the robot, and if this leads to better robotic
	control.

\end{itemize}


\subsection{Definition of inclusion criteria}
\label{sect|inclusion-criteria}

Every robotic system has, implicitely or not, some knowledge representation
systems. It may range from a simple state vector to an explicit symbolic
knowledge base.  This survey focuses on the right end of this spectrum:
symbolic systems, suited for abstract reasoning.

Besides, we have decided to restraint the set of systems to those actually
implemented on robots, and used in semantic-rich environments (\ie , dynamic,
partially unknown environments with a large range of different entities which
may have interactions). The typical scenario that would involve such robots is
a service robot in a human-friendly environment like a kitchen.

More precisely, we have established the following criteria to select the
knowledge representation systems to include in this survey. Those systems must:

\begin{itemize}
	\item  Run on \emph{service robot} (robots that interact with objects in a
	semantic-rich environment),

	\item  Ground their knowledge in the physical world (physically embedded),
	\begin{itemize}
		\item  Able to \emph{resolve entities}
		\item  Able to automatically create new object instances
	\end{itemize}

	\item  Be able to merge different knowledge modalities,
	\item  Be endowed with on-line, dynamic reasoning (not just a static
	dictionary)

\end{itemize}

\subsection{Evaluation of a knowledge representation system}
\label{sect|evaluation}


\subsubsection{Desirable features of a knowledge representation system}
\label{sect|evaluation-literature}

Several authors from fields that are connex to the robotics have previously listed desirable features of systems aiming at cognitive abilities as rich as possible.

\paragraph{McCarthy: Towards human-level AI}

In~\cite{McCarthy2007}, McCarthy lists the challenges he identifies on the road
to a \emph{human-level AI}.

\begin{itemize}

	\item the ability to \emph{"operate successfully in the common sense
	informatic situation"},

	\item the necessity of relying on mathematical logic, as the most fruitful
	formalism for machine intelligence,

	\item the ability to deal with \emph{approximate concepts and approximate
	theories} (that would include representing them, and reasoning with them),

	\item nonmonotonic reasoning \fxfatal{give here a clear example},

	\item what McCarthy calls \emph{Elaboration Tolerance}, \ie, the ability to
	extend \emph{on requirement} the closed domain of interpretation for a
	given assertion,

	\item the ability to formalize and reason about contexts,

	\item reasoning about events, and in particular, actions,

	\item the capacity of introspection,

	\item and finally, he points the issue of giving computer the right
	heuristics for decision making.

\end{itemize}

Knowledge representation systems are concerned by most, if not all, of this
points.

\paragraph{Natural language processing in situated context}

In~\cite{Roy2005}, Roy and Reiter summarize what they see as the main
challenges to be tackled: cross-modal representation systems, association of
words with perceptual and action categories, modeling of context, figuring out
the right granularity of models, integrating temporal modeling and planning,
ability to match past (learned) experiences with the current interaction and
ability to take into account the human perspective.

\subsubsection{Pyschological tests}
\label{sect|evaluation-tests}

\begin{itemize}
	\item Language comprehension tests: Token Test~\cite{DiSimoni1978}
\end{itemize}

\subsection{Methodology}
\label{sect|methodology}


\fxnote{Mention here that each team was proposed to complete/amend the description of their
system}

%%%%%%%%%%%%%%%%%%%%%%%%%%%%%%%%%%%%%%%%%%%%%%%%%%%%%%%%%%%%%%%%%%%%%%%%%%%%%%%%%%%%%%%%%%%
\section{Comparison criteria}
\label{sect|comparison-criteria}

\subsection{Intrinsic features}
\label{sect|intrinsic-features}

\subsubsection{Expressiveness}
\label{sect|expressiveness}

\paragraph{Logics formalism}

Which logic formalism (DL, 2nd order\ldots{})

On modal logics, see the remark of McCarthy, in \cite{McCarthy2007}, section 3

\paragraph{Open World and Close World Assumptions}

\fixme{cf Levesque~\cite{Levesque2007}, section 24.3.2}

\paragraph{(non) monotonic reasoning}

\paragraph{General knowledge + exception to this knowledge (blue sky/white sky)}

\paragraph{Representation of uncertainty}

\paragraph{Representation of time}

\paragraph{Representation of change}

E.g. \emph{"The pancake dough disappears into a pancake"}
 
\paragraph{Context and Microtheories}
Explicit modeling of context of knowledge / domain of validity

\subsubsection{Presupposition accomodation}
\label{sect|presupposition-accomodation}

\emph{Presupposition accomodation} is the ability to acquire and represent facts that are not
grounded into perception. A typical example is a human telling the robot that
\emph{"there is big blue sphere behind you!"}.

\subsubsection{Lazy evaluation}
\label{sect|lazy-evaluation}


\subsubsection{Introspection}
\label{sect|introspection}
Support for introspection?

\subsubsection{Prediction and projection tasks}
\label{sect|prediction-projection}

Levesque~\cite{Levesque2007} distinguish two main tasks, the \emph{projection task} and the \emph{legality task}.

\paragraph{Projection task}: determining whether or not some condition while hold after a sequence of actions.

\paragraph{Legality task}: determining whether a sequence of action can be performed starting in some initial state.

\subsubsection{Learning}
\label{sect|learning}


\subsection{Strategies to deal with the physical world}

\subsubsection{Knowledge acquisition}
\label{sect|knowledge-acquisition}

\paragraph{Perception}
\paragraph{Interaction}
\paragraph{External sources (Web, upper ontologies, ...)}
\paragraph{Learning}

\subsubsection{Grounding/anchoring strategies}
\label{sect|grounding}

\subsubsection{Ability to automatically create new object instances}
\label{sect|new-instances}

\subsubsection{Ability to merge modalities}
\label{sect|modalities-merging}

\subsection{Integration in a larger robotic architecture}
\label{sect|integration-robot}

\subsubsection{Integration with executive layers}
\label{sect|integration-executive-layers}

\paragraph{Events}

\paragraph{Relation to task planning}

\subsubsection{Interaction with humans}
\label{sect|hri}

Both as developers (KRS as a tool to make robot knowledge explicit to the
robot designer) and end-users (like natural language processing,...)

\subsubsection{Scalability and responsiveness}
\label{sect|scalability}


\subsection{Underlying knowledge model}

\begin{itemize}
	\item  Which underlying knowledge (\emph{common-sense}, \emph{upper knowledge}\ldots{})
	\begin{itemize}
		\item  top-down approach?
	\end{itemize}

\end{itemize}



%%%%%%%%%%%%%%%%%%%%%%%%%%%%%%%%%%%%%%%%%%%%%%%%%%%%%%%%%%%%%%%%%%%%%%%%%%%%%%%%%%%%%%%%%%%
\section{Surveyed systems}

Table \ref{table|surveyed-systems} presents the fifteen knowledge representation systems surveyed
in this article.

This section briefly presents each of them.

\begin{landscape}
\begin{table}
\begin{center}
\rowcolors{2}{lightgray}{codegray}

%\begin{tabularx}{\textheight}{p{2cm}p{4cm}p{3cm}p{4cm}p{2.3cm}p{2cm}p{2cm}}
\begin{tabular}{p{2.7cm}p{4cm}lp{2.5cm}p{4cm}lp{1.5cm}}
\hiderowcolors
{\bf Project} & {\bf Authors (Institution)} & {\bf Project homepage} & {\bf Programming language} & {\bf Knowledge model} & {\bf Reasoner} & Main reference \\
\hline
\showrowcolors
{\sc KnowRob} & Tenorth (TU Munich) & \url{http://ias.in.tum.de/kb/wiki} & {\sc Prolog} & {\sc Prolog} + OWL-DL & Custom ({\sc Prolog}) & \cite{Tenorth2009a} \\
ORO & Lemaignan \par (LAAS-CNRS) & \url{oro.openrobots.org} & {\sc Java} & OWL-DL ({\sc Jena}) & {\sc Pellet} & \cite{Lemaignan2010} \\
PEIS KR\&R & Daoutis, Coradeshi, Loutfi, Saffiotti \par (Örebro Univ.) & \url{www.aass.oru.se/~peis} & {\sc C}, {\sc CycL} & CycL (1st and 2nd order logics, modal logics) & & \cite{Daoutis2009} \\
DY-KNOW & Heintz, Dowerty \par (Linköping Univ.) & & & & & \cite{Heintz2004} \\
OMKRF & Suh et al. \par (Hanyang Univ.) & & & & & \cite{Suh2007} \\
GSM & Mavridis, Roy \par (MIT MediaLab) & & & & & \cite{Mavridis2006} \\
NKRL & Zarri et al. \par (Paris Est Créteil Univ.) & & NKRL & & & \cite{Sabri2011} \\
 & Wrighteagle \\
 & Vincze \\
 & (Bielfeld Univ.) \\
ARMAR & Schmidt-Rohr (Karlsruhe TH) \\
 & (Saarbrücken) \\
 & Hertzberg (Osnabrück Univ.) \\
 & (DFKI Bremen) \\
 (based on {\sc KnowRob} & (JSK) \\

\hline

%\end{tabularx}
\end{tabular}
\end{center}
\caption{List of surveyed systems}
\label{table|surveyed-systems}
\end{table}
\end{landscape}

\subsection{KnowRob}
\label{sect|knowrob}

\subsection{ORO}
\label{sect|oro}

\subsection{PEIS KR\&R}
\label{sect|peis-ecology}


{\sc PEIS Ecology}~\cite{Saffiotti2005} is a software \emph{ecosystem} that aim to binds autonomous
robotics with ambient intelligence (network of sensors). \emph{PEIS} stands for
\emph{Physically Embedded Intelligent System}: every robots or intelligent
device in the environment is abstracted as a PEIS.

Each PEIS physical component is running a \emph{PEIS Kernel} instance. Communication
between instance relies on a custom P2P communication protocol.

The PEIS architecture allows for adding new abilities through software components sharing the common \emph{tuple space}.

We survey here the semantic layer~\cite{Daoutis2009}, referred as \emph{PEIS KR\&R}, that includes symbolic representation and reasoning.

% More in details:
% - object identification based on viewpoint independent SIFT features
% - formalized anchoring system that explicitely match percieved attributes to predicates
% - Cyc predicates
% - ground 12 colors, based on a paper on color perception. Could be useful for us.
% - idem, they cite a paper on what spatial relations to compute
% - location of objects based on a previously provided semantic map (but not much on this semantic map)
% - two "memories": the robot memory stores the current list of percieved objects ; the archive memory stores what is not percieved anymore
% - uses directly Cyc (ie, 250 000 common sense concepts...), via CycL language -> 2nd and higher order logics (quantification over predicates, functions, etc)
% Remark: using 2nd order logic (ie meta statements), it would be easy to store the knowledge of each agent
% - disambiguation in concept name by asking human to decide amongst all concepts known by Cyc
% - template based natural language
% - experiment conducted in a "smart" indoor environmement + simple robot

\subsubsection{Intrinsic features}
\label{sect|peis-intrinsic-features}

\paragraph{Expressiveness} The PEIS Knowledge representation system relies on
the {\sc ResearchCyc} and {\sc CycL} language to represent knowledge. The {\sc CycL} language
allows to represent first order logic sentences and has extensions for modal logics and higher order logics.

\fxfatal{Is modal logics and higher order logics actually used in PEIS?} 

As a system relying on {\sc CycL}, contexts can be expressed as
\emph{microtheories}: the truth or falsity of a set of statement depends of the
\emph{microtheory} in which these statements are evaluated.

\fxfatal{OWA/CWA?}

\subsubsection{Anchoring strategies}
\label{sect|peis-anchoring}

\begin{figure}
	\centering
	\includegraphics[width=0.9\columnwidth]{peis-architecture.pdf}
	\caption{The PEIS knowledge representation system, taken from~\cite{Daoutis2009}}
	\label{fig|peis-archi}
\end{figure}

The PEIS KR\&R system is deeply integrated to the general PEIS Ecology
\emph{smart} environment. Figure~\ref{fig|peis-archi} gives an overview of the
interactions between PEIS knowledge processing layers.

\paragraph{Knowledge Acquisition} The primary source for knowledge acquisition
is perception.  The PEIS ecosystem provides a SIFT-based object recognizer used
in conjunction with ceiling cameras for object localization.  Other perceptual
modalities are available (like human tracking, ambient environment monitoring).

A template-based natural language parsing system may also be used to add new
assertions to the system.

\paragraph{Anchoring} Daoutis et al. formalize the issue of anchoring as
finding a \emph{predicate grounding relation} $g \subseteq \mathcal{P} \times
\Phi \times D(\Phi)$, where $\mathcal{P}$ is a set of predicate symbols, $\Phi$
a set of percept's attributes, and $D(\Phi)$ the domain of these attributes.

In the current implementation, object category (returned by the SIFT
classifier), color, location, spatial relations (both topological -- \emph{at},
\emph{near} -- and relative to the robot -- \emph{left}, \emph{behind}, etc.)
and visibility are the five classes of extracted attributes.

\subsubsection{Integration in the robot architecture}
\label{sect|peis-integration}

The PEIS framework offers through the \emph{PEIS middleware} a practical way to
insert a new component into the shared \emph{tuple space}.  Thus, the KR\&R
module can be seemlessly integrated into the PEIS ecosystem.

\subsubsection{Notable experiments}
\label{sect|peis-expe}

\subsection{NKRL}
\label{sect|nkrl}

\emph{NKRL} stands for \emph{Narrative Knowledge Representation Language}.
While this language is developped since a long time by Zarri~\cite{Zarri1997,
Zarri2008}, recent research direction include application to the robotic
field~\cite{Sabri2011}. NKRL is not {\it per-se} a knowledge representation
system, as it is primarily a language. However, it is used as the
representation and reasoning mechanism for robots by Sabri et al.

\subsubsection{Intrinsic language features}
\label{sect|nkrl-intrinsic-features}

\paragraph{Expressiveness}

\subsubsection{Integration with physical world and in the robot architecture}
\label{sect|nkrl-integration}

...seem to be mostly WIP...

\begin{table}
\begin{center}
\rowcolors{2}{lightgray}{codegray}

\begin{tabular}{ll}
\hiderowcolors
{\bf Project} & {\bf Common-sense knowledge source} \\
\hline
\showrowcolors
{\sc KnowRob} & {\sc OpenCyc}, processed web content, custom OWL-DL ontology \\
ORO & {\sc OpenCyc}, custom OWL-DL ontology \\
PEIS Ecology & {\sc ResearchCyc} \\
DY-KNOW & \\
OMKRF & \\
GSM &  \\
NKLR &  None \\
Wrighteagle & \\
Vincze & \\
(Bielfeld) & \\
ARMAR &  \\
(Saarbrücken) & \\
Hertzberg (Osnabrück) & \\
(DFKI Bremen) & \\
(JSK) & \\

\hline

\end{tabular}
\end{center}
\caption{Underlying knowledge sources for each project}
\label{table|knowledge-sources}
\end{table}


%%%%%%%%%%%%%%%%%%%%%%%%%%%%%%%%%%%%%%%%%%%%%%%%%%%%%%%%%%%%%%%%%%%%%%%%%%%%%%%%%%%%%%%%%%%
\section{Conclusion}
\label{sect|conclusion}

\subsection{Main groups}

\subsection{What is not successfully covered by current systems}

\subsection{Future challenges}
\label{sect|future-challenges}

%%%%%%%%%%%%%%%%%%%%%%%%%%%%%%%%%%%%%%%%%%%%%%%%%%%%%%%%%%%%%%%%%%%%%%%%%% 
\section*{Acknowledgements} 

%%%%%%%%%%%%%%%%%%%%%%%%%%%%%%%%%%%%%%%%%%%%%%%%%%%%%%%%%%%%%%%%%%%%%%%%%% 

\bibliographystyle{ieeetr}
\bibliography{biblio}


\end{document}
